% PLANTILLA APA7
% Creado por: Isaac Palma Medina
% Última actualización: 25/07/2021
% @COPYLEFT

% Fuentes consultadas (todos los derechos reservados):  
% Normas APA. (2019). Guía Normas APA. https://normas-apa.org/wp-content/uploads/Guia-Normas-APA-7ma-edicion.pdf
% Tecnológico de Costa Rica [Richmond]. (2020, 16 abril). LaTeX desde cero con Overleaf (1 de 3) [Vídeo]. YouTube. https://www.youtube.com/watch?v=kM1KvHVuaTY Weiss, D. (2021). 
% Formatting documents in APA style (7th Edition) with the apa7 LATEX class. https://ctan.math.washington.edu/tex-archive/macros/latex/contrib/apa7/apa7.pdf @COPYLEFT

%+-+-+-+-++-+-+-+-+-+-+-+-+-++-+-+-+-+-+-+-+-+-+-+-+-+-+-+-+-+-++-+-+-+-+-+-+-+-+-+

% Preámbulo
\documentclass[stu, 12pt, letterpaper, donotrepeattitle, floatsintext, natbib]{apa7}
\usepackage[utf8]{inputenc}
\usepackage{comment}
\usepackage{marvosym}
\usepackage{graphicx}
\usepackage{float}
\usepackage[normalem]{ulem}
\usepackage[spanish]{babel} 
\selectlanguage{spanish}
\useunder{\uline}{\ul}{}
\newcommand{\myparagraph}[1]{\paragraph{#1}\mbox{}\\}

% Portada
\thispagestyle{empty}
\title{\Large Título del documento PERRAS pipipipii}
\author{Salazar Arias, Luis Santiago de Blass \\Bravo Ocampo, David Aquiles \\Flores Grados, Andrea Hilda\\Narvaez Garriazo, Cristel Margarita} % (autores separados, consultar al docente)
% Manera oficial de colocar los autores:
%\author{Autor(a) I, Autor(a) II, Autor(a) III, Autor(a) X}
\affiliation{\Huge\textbf{Universidad San Ignacio de Loyola}}
\course{\normalsize{Código del curso: Nombre del curso}}
\professor{Nombre del docente}
\duedate{Fecha}
\begin{document}
\maketitle


% Índices
\pagenumbering{roman}
    % Contenido
\renewcommand\contentsname{\largeÍndice}
\tableofcontents
\setcounter{tocdepth}{2}
\newpage
    % Fíguras
\renewcommand{\listfigurename}{\largeÍndice de fíguras}
\listoffigures
\newpage
    % Tablas
\renewcommand{\listtablename}{\largeÍndice de tablas}
\listoftables
\newpage

% Cuerpo
\pagenumbering{arabic}

\section{\large Título I}
\noindent \maskCitet{cervantes1999}\\
En un lugar de la Mancha, de cuyo nombre no quiero acordarme, no ha mucho tiempo que vivía un hidalgo de los de lanza en astillero, adarga antigua, rocín flaco y galgo corredor.
\subsection{Título II} 
Una olla de algo más vaca que carnero, salpicón las más noches, duelos y quebrantos los sábados, lantejas los viernes, algún palomino de añadidura los domingos, consumían las tres partes de su hacienda.
\subsubsection{Título III}
El resto della concluían sayo de velarte, calzas de velludo para las fiestas, con sus pantuflos de lo mesmo, y los días de entresemana se honraba con su vellorí de lo más fino.
\paragraph{Título IV}
Tenía en su casa una ama que pasaba de los cuarenta, y una sobrina que no llegaba a los veinte, y un mozo de campo y plaza, que así ensillaba el rocín como tomaba la podadera.
\myparagraph{Título IV ii}
Frisaba la edad de nuestro hidalgo con los cincuenta años; era de complexión recia, seco de carnes, enjuto de rostro, gran madrugador y amigo de la caza. 
\subparagraph{Título V}
Quieren decir que tenía el sobrenombre de Quijada, o Quesada, que en esto hay alguna diferencia en los autores que deste caso escriben; aunque por conjeturas verosímiles se deja entender que se llamaba Quijana.

\newpage
% Referencias
\renewcommand\refname{\large\textbf{Referencias}}
\bibliography{mibibliografia}

\end{document}